\documentclass[11pt, a4paper, twoside]{montblanc2}
\usepackage[T1]{fontenc}
\usepackage{url}
\usepackage{color}
\usepackage{xspace}

\def\lua{\textsc{Lua}\xspace}

\begin{document}
\devnum{[5.11]}
\title{[MAQAO in BOAST]}
\version{[0.1]}
\deadline{[2017/01/16]}
\level{[PU]}
\nature{[O]}
\authors{Olivier Aumage (Inria) and Brice Videau (CNRS)}
\contributors{} % {Name (PARTNER), Name (PARTNER), Name (PARTNER)}
\reviewers{} % {Name (PARTNER), Name (PARTNER), }
\keywords{[analysis, autotuning, performance]}

\maketitle

\begin{changelog}
\change{0.1}{Initial version of D5.11}
\end{changelog}

\frontmatter

\begin{executive}
  \todo{executive summary}
\end{executive}

\section{Introduction}
\todo{intro}

\section{Context}
\todo{context}

\subsection{BOAST}
\begin{itemize}
  \item General presentation of BOAST
  \item Purpose, capabilities, usage
\end{itemize}

\subsection{MAQAO}
\begin{itemize}
  \item General presentation of MAQAO
  \item Purpose, capabilities, usage
\end{itemize}

MAQAO, the Modular Assembly Quality Analyzer and Optimizer, is a performance analysis and processing 
tool working at the level of compiled, binary code. It was initially developed at the University of 
Versailles in France, and has been partially developed at Inria in Bordeaux as well since around 
2010. While primarily focused on Intel processors (x86-64, xeon phi, and even Itanium), it was 
ported on the ARM architecture as part of Mont-Blanc~2 task D5.3 work, and delivered as 
Deliverable~D5.4. The MAQAO tool presents itself as an extensible framework for binary code 
processing. It is a made of a C-language low-level core set of libraries and a high-level set of 
wrappers making the framework scriptable in the language \lua using the embedded interpretor. 

The core libraries of MAQAO enable the fundamental operations to disassemble a binary file, modify 
its assembly instructions (operation designated as 'patching'), and re-assemble the modified binary. 
Possible instruction modifications include moving/inserting/suppressing code blocks, inserting 
function calls, or operations designated as 'instrumentation' where some sets of instructions are 
both moved and modified such as a function call is made every time such instructions are called 
before executing the instrumented instructions themselves. Core services also include foundational 
capabilities such as building call graphs and control flow graphs from the disassembled binary 
functions.

A set of \lua-to-C wrapper routines implement the transition between the C~core and the Lua 
services. These routines enable to manipulate relevant assembly objects such as binary files, 
functions, basic instruction blocks, individual instructions, instruction operands, loops and 
labels. 

On top of those transitional routines, special-purpose processing modules can be implemented in Lua 
to perform high level operations such as analysis, tracing and hinting. In particular, tracing 
involves instrumenting every memory reference in the studied kernel with calls to the memory access 
accounting routine in the MAQAO Trace Library (MTL) which monitor every memory address referenced 
within the kernel routine during an execution, and generate a compressed trace of the memory 
references. The trace can then be further processed to output memory access patterns as human 
readable algebraic expressions.



\section{Integration}
  \todo{integration}

  \subsection{Big Picture}
  \subsection{Usage}

\section{Implementation}
  \todo{implementation}

\subsection{Data Exchange}

\begin{itemize}
  \item Requirements
  \item YAML structured data
  \item BOAST --> MAQAO
  \item MAQAO --> BOAST
\end{itemize}

\subsection{Kernel generation}

\begin{itemize}
  \item Requirements
    \begin{itemize}
      \item Executable kernel
      \item No position-independent binary
      \item Dwarf debug info
    \end{itemize}
  \item Kernel wrapper
\end{itemize}

\subsection{Kernel instrumentation}

\begin{itemize}
  \item Principle
    \begin{itemize}
      \item Memory access asm instruction rewriting
    \end{itemize}
  \item Requirements
    \begin{itemize}
      \item System V ABI + ARM ABI compliance
    \end{itemize}
  \item ASM patching implementation
  \item Lua high-level instrumentation implementation
  \item MAQAO tracing library
\end{itemize}

\subsection{Kernel analysis framework}

\begin{itemize}
  \item Principle
  \item C low-level framework
  \item Lua high-level framework
  \item SIMD analyzer example
\end{itemize}

\section{Testcases examples}
  \todo{implementation}

  \subsection{Vector Addition}
\begin{itemize}
  \item Basic vector addition kernel
\end{itemize}

  \subsection{BigDFT Filter}
\begin{itemize}
  \item Mont-Blanc 2 Application BigDFT kernel
\end{itemize}

\section{Conclusion and Future Work}
\todo{conclusion}

\end{document}


% vim: set spell ft=tex fo=aw2t expandtab sw=2 tw=100:
