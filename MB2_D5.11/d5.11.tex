\documentclass[11pt, a4paper, twoside]{montblanc2}
\usepackage[T1]{fontenc}
\usepackage{url}
\usepackage{color}
\usepackage{xspace}

\def\cmake{\textsc{CMake}\xspace}
\def\lua{\textsc{Lua}\xspace}
\def\dd{\textsc{DwarfDump}\xspace}
\def\elfutils{\textsc{ElfUtils}\xspace}

\begin{document}
\devnum{[5.11]}
\title{[MAQAO in BOAST]}
\version{[0.1]}
\deadline{[2017/01/16]}
\level{[PU]}
\nature{[O]}
\authors{Olivier Aumage (Inria) and Brice Videau (CNRS)}
\contributors{} % {Name (PARTNER), Name (PARTNER), Name (PARTNER)}
\reviewers{} % {Name (PARTNER), Name (PARTNER), }
\keywords{[analysis, autotuning, performance]}

\maketitle

\begin{changelog}
\change{0.1}{Initial version of D5.11}
\end{changelog}

\frontmatter

\begin{executive}
  \todo{executive summary}
\end{executive}

\section{Introduction}
\todo{intro}

\section{Context}
\todo{context}

\subsection{BOAST}
\begin{itemize}
  \item General presentation of BOAST
  \item Purpose, capabilities, usage
\end{itemize}

\lstdefinestyle{Bash}{
  language=Bash,
  basicstyle=\small\sffamily,
  frame=single,
  rulecolor=\color{bluelst},
  commentstyle=\color{greenlst}\textit,
  keywordstyle=\bfseries,
  breaklines=true,
  numbers=left,
  numberstyle=\tiny,
  numbersep=5pt,
  tabsize=4
}

\subsubsection{Presentation}

BOAST is an automatic performance tuning framework aiming at meta-programming and optimizing computing kernels.
BOAST has been extensively described in the Mont-Blanc 2 deliverable 5.5~\cite{tichadou15} and thus will only be briefly described here.
For a more in-depth presentation, reader should refer to the afore mentioned document.

The primary goal of BOAST is to bring performance portability to critical parts of high performance computing applications.
Those \emph{computing kernels} are well-defined parts of an application that are compute or memory intensive and represent a significant part of the computing time.
Their inputs are usually limited in complexity, thus those kernels are a prime target for optimization.

Unfortunately, the variety of platforms an HPC code can encounter keeps growing, and a lot of time is spent optimizing computing kernels for new architectures.
BOAST attempts to solve this problem by offering the kernel developer several tools:
\begin{itemize}
  \item an Embedded Domain Specific language to describe computing kernels and their possible optimizations;
  \item a code generation engine that can output the kernel in several languages used in the HPC community: C, FORTRAN, CUDA and OpenCL;
  \item a runtime to select the versions to test, build them, execute them and test their performance and accuracy.
\end{itemize}


\subsubsection{Usage}


The usage of BOAST will be illustrated using the following Ruby/BOAST listing.
This example shows how to load BOAST, change it's environment, define a simple vector addition kernel, execute it using different languages, and check the accuracy of the obtained results.

\lstset{style=BOAST}
\begin{lstlisting}
require 'BOAST'
include BOAST

set_array_start(0)
set_default_real_size(4)

def vector_add
  n = Int( "n", :dir => :in)
  a = Real("a", :dir => :in,  :dim => [ Dim(n) ] )
  b = Real("b", :dir => :in,  :dim => [ Dim(n) ] )
  c = Real("c", :dir => :out, :dim => [ Dim(n) ] )
  p = Procedure("vector_add", [n,a,b,c]) {
    decl i = Int("i")
    expr = c[i] === a[i] + b[i]
    if (get_lang == CL or get_lang == CUDA) then
      pr i === get_global_id(0)
      pr expr
    else
      pr For(i,0,n-1) {
        pr expr
      }
    end
  }
  return p.ckernel
end

n = 1024*1024
a = NArray.sfloat(n).random
b = NArray.sfloat(n).random
c = NArray.sfloat(n)

c_ref = a + b
epsilon = 10e-15

[FORTRAN, C, CL, CUDA].each { |l|
  set_lang( l )
  puts "#{get_lang_name}:"
  k = vector_add
  puts k
  c.random!
  k.run(n, a, b, c, global_work_size: [n,1,1], local_work_size: [32,1,1])
  diff_max = (c_ref - c).abs.max
  raise "Error: max error too big: #{diff_max}!" if diff_max > epsilon
}
puts "Success!"
\end{lstlisting}

\paragraph{Loading BOAST} is done on lines 1 and 2.
It loads the BOAST framework in its environmental configuration.
This configuration can be changed through configuration files or environment variables.
Here, this configuration is changed on lines 4 and 5 to set the default real variable size to 4 bytes and the first index of arrays to 0 (C like arrays).

\paragraph{Kernel definition} is done on lines 7 to 25.
The definition is enclosed in a procedure that will be called in different contexts, yielding different kernel versions.
First, the parameters of the kernel are defined on lines 8 to 11, and stored in ruby variables.
Those parameters have a direction (similar to FORTRAN intent). The \emph{Real} parameters have a dimension, meaning they are arrays, in this case of length \emph{n}.
The \emph{Procedure} is declared on line 12 by giving its name and parameter list.
It is associated a block of code that spans from line 12 to 23 and constitute its body.
A local integer variable \emph{i} is defined on line 12 and declared using the keyword \emph{decl}.
On line 14, an assignment expression is saved for later use in the ruby variable \emph{expr}.
This expression computes one element of the result array.
The difference between a ruby assignment (using the \emph{=} operator) and a BOAST assignment (using the \emph{===} operator) has to be noted.
On lines 15 to 22 we can see some BOAST meta-programming.
Here, based on the BOAST \emph{lang} state, the code will select either a GPU or CPU implementation.
The GPU implementation obtains an index of the element to process (line 16) and stores it into the \emph{i} variable.
This is printed using the \emph{pr} keyword, meaning that this line is to appear in the generated source.
The expression we saved earlier is then printed (line 17).
The CPU implementation prints a for loop that will process all the elements of the arrays.
We can see here an example, though far-fetched, of code factoring.
The line 24 returns the computing kernel object corresponding to the printed \emph{p} procedure.
The procedure \emph{vector\_add} is automatically set as the entry point of the computing kernel.

\paragraph{Kernel execution} is done on lines 27 to 48.
Line 27 to 30 define input values for our kernel.
Three of these objects are numerical arrays of size \emph{n}, two of them (input ones) are initialized with random values.
A reference result is computed on line 32.
An epsilon is defined to check the accuracy of further computed results (line 33).
From line 35 to 44 the kernel will be generated for each target language supported by BOAST.
To do this we first change the BOAST \emph{lang} state and print it on the standard output (lines 36 and 37).
The \emph{vector\_add} ruby procedure is then called, which generates a new instance of the computing kernel, adapted to the chosen language (line 38).
The generated code is then printed to the standard output.
The output array is set randomly and then the computing kernel is called through the \emph{run} method  (line 44).
Compilation is launched implicitly here, it could also be requested explicitly by calling the \emph{build} method.
First, the kernel takes positional arguments that correspond to its parameters.
The following parameters are named options that are ignored in \emph{C} and \emph{FORTRAN} but interpreted in \emph{OpenCL} and \emph{CUDA}.
Once the kernel has finished executing we compare the output \emph{c} to the computed reference (line 42) and raise an error should the maximum discrepancy be superior to \emph{epsilon}.

\paragraph{Installation} of BOAST is made through the simple following command provided ruby >= 1.9.3 and its development files are installed.

\lstset{style=Bash}
\begin{lstlisting}
gem install --user-install BOAST
\end{lstlisting}



\subsection{MAQAO}
\begin{itemize}
  \item General presentation of MAQAO
  \item Purpose, capabilities, usage
\end{itemize}

\subsubsection{Presentation}
MAQAO, the Modular Assembly Quality Analyzer and Optimizer, is a performance analysis and processing 
tool working at the level of compiled, binary code. It was initially developed at the University of 
Versailles in France, and has been partially developed at Inria in Bordeaux as well since around 
2010. While primarily focused on Intel processors (x86-64, xeon phi, and even Itanium), it was 
ported on the ARM architecture as part of Mont-Blanc~2 task D5.3 work, and delivered as 
Deliverable~D5.4. The MAQAO tool presents itself as an extensible framework for binary code 
processing. It is a made of a C-language low-level core set of libraries and a high-level set of 
wrappers making the framework scriptable in the language \lua using the embedded interpretor. 

The core libraries of MAQAO enable the fundamental operations to disassemble a binary file, modify 
its assembly instructions (operation designated as 'patching'), and re-assemble the modified binary. 
Possible instruction modifications include moving/inserting/suppressing code blocks, inserting 
function calls, or operations designated as 'instrumentation' where some sets of instructions are 
both moved and modified such as a function call is made every time such instructions are called 
before executing the instrumented instructions themselves. Core services also include foundational 
capabilities such as building call graphs and control flow graphs from the disassembled binary 
functions.

A set of \lua-to-C wrapper routines implement the transition between the C~core and the Lua 
services. These routines enable to manipulate relevant assembly objects such as binary files, 
functions, basic instruction blocks, individual instructions, instruction operands, loops and 
labels. 

On top of those transitional routines, special-purpose processing modules can be implemented in Lua 
to perform high level operations such as analysis, tracing and hinting. In particular, tracing 
involves instrumenting every memory reference in the studied kernel with calls to the memory access 
accounting routine in the MAQAO Trace Library (MTL) which monitor every memory address referenced 
within the kernel routine during an execution, and generate a compressed trace of the memory 
references. The trace can then be further processed to output memory access patterns as human 
readable algebraic expressions.

\subsubsection{Usage}

The MAQAO framework version developed in Bordeaux is very much targeted at an experimented audience, 
and, as a note of warning, it may lack the user friendliness commonly found in mature academic and 
commercial programming tools. It is aimed at expert users desirous to dig further into the binary 
kernel code produced by their compiler, and build their own, custom analysis strategies.
This section presents MAQAO from a functional point-of-view, and specify usage for the most common
operations.

\paragraph{Requirements and Installation}

MAQAO's C language core mandatorily requires \cmake version~2.8.8 or above, a C compiling toolchain 
and Make. The \lua layers directly make use of the embedded \lua distribution. Moreover, MAQAO 
analysis scripts may leverage the command-line tool \dd to process debugging informations and 
symbols in compiled objects files when available, while \dd itself relies on the availability of an 
\elfutils distribution. Supported binary objects and executables to be processed by MAQAO must 
follow the ELF format specification and System V general ABI specification, and be generated for the 
ARM architecture, while optional debugging information must follow the DWARF-2 format. Support for 
ARM's Thumb instruction sets is limited to disassembly only.

Building MAQAO involves running \cmake followed by the \verb|make| command after untaring the 
distribution or checking it out from the development repository:

\begin{verbatim}
$ tar zxf maqao.tar.gz
$ cd MAQAO
$ mkdir build
$ cd build
$ cmake -DARCHS=arm -DSTRIP=false ..
$ make
\end{verbatim}

No install step is required. MAQAO's executable is produced in \verb|MAQAO/bin| and MAQAO's 
libraries are placed in the \verb|MAQAO/lib| directory. MAQAO's \verb|/bin| and \verb|/lib|
directories should be added to \verb|PATH| and \verb|LD_LIBRARY_PATH| environment variables 
respectively. Environment variable \verb|MAQAO| should be set to MAQAO's top directory for 
convenience, and is assumed to be set accordingly in the remainder of the document.

\paragraph{Disassemble}

Disassembling and low level assembly management is handled by MAQAO's Madras module. Every MAQAO 
module is accessed by specifying its name as first argument of MAQAO. Module's arguments are then 
prepended after the module name. Binary disassembled listing can be obtained using Madras as 
follows:

\begin{verbatim}
$ maqao madras -d <BINARY>
...
bf4c <s1111>:
 [...] f0 45 2d e9  0xbf4c push   {r4, r5, r6, r7, r8, sl, lr}; [...]
 [...] 02 8b 2d ed  0xbf50 vpush  d8; [...]
 [...] 14 d0 4d e2  0xbf54 sub    sp, sp, #20 ; [...]
 [...] 94 0d 0f e3  0xbf58 movw   r0, #64916 ; [...]
 [...] 01 00 40 e3  0xbf5c movt   r0, #1 ; [...]
...
\end{verbatim}

\paragraph{Instrument}

Instrumentation is the process of inserting probes for a set of specific
assembler instructions such that every time such an instruction is encountered
during execution, some accounting probe function can be called to account for
that instruction. Instrumentation in MAQAO is provided by the Memory module,
to target memory referencing instructions, such as to extract memory access
patterns at run-time. The Memory module itself calls the Madras module under the
hood to perform the low-level assembly manipulation involved.

\begin{verbatim}
$ maqao memory -i -bin=<BINARY> -f=<FUNCTION_NAME> -m=unicore -tp==<TEMP_DIR>
\end{verbatim}

Argument \verb|-i| requests instrumenting. Argument \verb|-bin| specifies the
binary executable to instrument. The MAQAO ARM port does not support
instrumenting position independent code (PIC) currently, thus the binary file
cannot be a shared library. Argument \verb|-f| specifies the function to
instrument. Argument \verb|-m| specifies the execution model. Finally, argument
\verb|-tp| specifies the directory in which the resulting instrumented binary
and its companion \lua meta-data file are generated. The actual instrumentations 
probe functions are trace recording routines from the MAQAO Trace Library (MTL).

Example below show diff between a uninstrumented binary listing fragment and the 
corresponding instrumented assembly code. Instruction \verb|vldr s14, [r1]| at 
address \texttt{0xbe78} (and other memory referencing instructions as well, such 
as the next one) is replaced by a branch to an inserted instruction block 
generically named \verb|@_patchmov_@|, generated by Madras to perform the memory 
access accounting.
\begin{footnotesize}
\begin{verbatim}
...
be78: ed917a00 vldr    s14, [r1]             | be78: ea15387a b       55a068 <@_patchmov_@>
be7c: ed537a01 vldr    s15, [r3, #-4]        | be7c: ea153899 b       55a0e8 <@_patchmov_@+
be80: ee777a27 vadd.f32        s15, s14, s15   be80: ee777a27 vadd.f32        s15, s14, s15
be84: ee171a90 vmov    r1, s15                 be84: ee171a90 vmov    r1, s15
...
\end{verbatim}
\end{footnotesize}

The assembly fragment below show part of the \verb|@_patchmov_@| block inserted 
by Madras. This block residing in the newly created \texttt{.madras.code} ELF 
section in the executable file is immediately preceded by another newly created 
section named \texttt{.madras.plt} containing the corresponding Procedure 
Linkage Table (PLT) containing the relocation entries for the indirect jumps to 
the MAQAO Trace Library routines.

\begin{footnotesize}
\begin{verbatim}
...
   >
   > Disassembly of section .madras.plt:
   >
   > 0055a044 <.madras.plt>:
   >   55a044: e3a0ca08 mov     ip, #32768      ; 0x8
   >   55a048: e59cc440 ldr     ip, [ip, #1088] ; 0x4
   >   55a04c: e5bcf038 ldr     pc, [ip, #56]!  ; 0x3
   >   55a050: e3a0ca08 mov     ip, #32768      ; 0x8
   >   55a054: e59cc440 ldr     ip, [ip, #1088] ; 0x4
   >   55a058: e5bcf03c ldr     pc, [ip, #60]!  ; 0x3
   >   55a05c: e3a0ca08 mov     ip, #32768      ; 0x8
   >   55a060: e59cc440 ldr     ip, [ip, #1088] ; 0x4
   >   55a064: e5bcf040 ldr     pc, [ip, #64]!  ; 0x4
   >
   > Disassembly of section .madras.code:
   >
   > 0055a068 <@_patchmov_@>:
   >   55a068: e52d0004 push    {r0}            ; (st

   >   55a0a8: e10f0000 mrs     r0, CPSR
   >   55a0ac: e92d47ff push    {r0, r1, r2, r3, r4, 
   >   55a0b0: ed2d8b10 vpush   {d8-d15}
   >   55a0b4: e24dd014 sub     sp, sp, #20
   >   55a0b8: e3000007 movw    r0, #7
   >   55a0bc: e1a02001 mov     r2, r1
   >   55a0c0: e52d2004 push    {r2}            ; (st
   >   55a0c4: e49d1004 pop     {r1}            ; (ld
   >   55a0c8: ebffffe0 bl      55a050 <__bss_end__+0
   >   55a0cc: e28dd014 add     sp, sp, #20
   >   55a0d0: ecbd8b10 vpop    {d8-d15}
   >   55a0d4: e8bd47ff pop     {r0, r1, r2, r3, r4, 
   >   55a0d8: e12ff000 msr     CPSR_fsxc, r0
   >   55a0dc: e49d0004 pop     {r0}            ; (ld
   >   55a0e0: ed917a00 vldr    s14, [r1]
   >   55a0e4: eaeac764 b       be7c <s111+0x60>
...
\end{verbatim}
\end{footnotesize}

The job of the code inserted in \verb|@_patchmov_@| is basically to:
\begin{enumerate}
\item \texttt{55a068..55a0b4:} Save the current program state (registers);
\item \texttt{55a0c8..55a0c8:} Call the relevant MTL accounting routine with the 
memory address of the instrumented memory reference, register \texttt{r1} here, 
in the case of instruction \texttt{0xbe78}, by jumping to the corresponding PLT 
entry and from there to the actual accounting routine;
\item \texttt{55a0cc..55a0dc:} Restore the program registers;
\item \texttt{55a0e0:} Execute the original, moved memory referencing 
instruction \verb|vldr s14, [r1]|;
\item \texttt{55a0e4:} Branch back to the main program at address 
  \texttt{0xbe7c}, immediately following the instrumented location 
  \texttt{0xbe78}, to continue program execution (which fortuitously happens to 
  be a memory reference also, and is thus instrumented as well).
\end{enumerate}

\paragraph{Trace}

Executing the resulting instrumented kernel produces a trace file containing a 
compressed representation of the target addresses of the memory accesses 
instruction encountered during execution. For each instruction instrumented, the 
flow of addresses captured is compressed on-the-fly using a lossless
algorithm name \emph{Nested Loop Recognition} (NLR), designed by Ketterlin and 
Clauss~\cite{ketterlin:nlr:cgo:2008}. The trace is stored using a compact text 
file format, not meant to be read as is by the programmer, as shown below.

\begin{footnotesize}
\begin{verbatim}
D 1 E 10 R 1 I 1 K 0 V  L P 39 L P 259 0 T 1 P 1262052 0 8 0 N N N I 2 K 0 V  L 
P 39 L P 259 0 T 1 P 3439840 0 8 0 N N N I 3 K 0 V  L P 39 L P 259 0 T 1 P 
3439852 0 8 0 N N N I 4 K 0 V  L P 39 T 1 P 3199305884 0 N N I 5 K 0 V  L P 39 T 
1 P 3199305888 0 N N I 6 K 0 V  L P 39 T 1 P 3199305892 0 N N I 7 K 0 V  L P 39 
T 1 P 3199305896 0 N N I 8 K 0 V  L P 39 L P 259 0 T 1 P 1262052 0 8 0 N N N I 9 
K 0 V  L P 39 L P 259 0 T 1 P 3439840 0 8 0 N N N I 10 K 0 V  L P 39 L P 259 0 T 
1 P 3439852 0 8 0 N N N 
\end{verbatim}
\end{footnotesize}

This intermediate format can however be processed by MAQAO's Memory module to 
produce a human-readable version of the NLR trace. The command below produces
the human-readable form of the trace from the compact trace and from the companion
\lua meta-data file generated during the instrumentation step.
 
\begin{verbatim}
$ maqao memory -d -t=<COMPACT_TRACE> -meta=<META_LUA_FILE> -tp=<TEMP_DIR>
\end{verbatim}

 This human-readable version shows, for each instrumented instruction, the
 corresponding thread id, the id of the loop containing the instruction as
 assigned by MAQAO, and the 'instrumented instruction' id also incrementally
 assigned by MAQAO. Multiple instances may exist when multiple threads invoke
 the instruction.

\begin{footnotesize}
\begin{verbatim}
Info: ################################################################################
Info: ## Volume for tid = 0 loopid = 21 iid = 0
Info: ################################################################################
Info: Instance 0
Info: ###############
for i0 = 0 to 39
  for i1 = 0 to 259
    val 0x1341e4 + 8*i1

 [...]
\end{verbatim}
\end{footnotesize}

Then, following the identification information, the successive values captured
by the instrumentation are represented by as a pseudo source-code with loops and
expressions. The expressions describe the memory addresses that have been
accessed, and depend on the surrounding loop counters and loop bounds, as
detected by the NLR algorithm. Expressions can however, only depend linearly on
the loop counters, and loop bounds only depend linearly on enclosing loops'
counters. The memory addresses referenced by an expression forms an union of
polytopes. Thus, the method captures the memory working-set as well as a
schedule for the memory accesses.

\paragraph{Analyze}

\paragraph{Extend and customize}

\section{Integration}
  \todo{integration}

  \subsection{Big Picture}
  \subsection{Usage}

\section{Implementation}
  \todo{implementation}

\subsection{Data Exchange}

\begin{itemize}
  \item Requirements
  \item YAML structured data
  \item BOAST --> MAQAO
  \item MAQAO --> BOAST
\end{itemize}

\subsection{Kernel generation}

\begin{itemize}
  \item Requirements
    \begin{itemize}
      \item Executable kernel
      \item No position-independent binary
      \item Dwarf debug info
    \end{itemize}
  \item Kernel wrapper
\end{itemize}

\subsection{Kernel instrumentation}

\begin{itemize}
  \item Principle
    \begin{itemize}
      \item Memory access asm instruction rewriting
    \end{itemize}
  \item Requirements
    \begin{itemize}
      \item System V ABI + ARM ABI compliance
    \end{itemize}
  \item ASM patching implementation
  \item Lua high-level instrumentation implementation
  \item MAQAO tracing library
\end{itemize}

\subsection{Kernel analysis framework}

\begin{itemize}
  \item Principle
  \item C low-level framework
  \item Lua high-level framework
  \item SIMD analyzer example
\end{itemize}

\section{Testcases examples}
  \todo{implementation}

  \subsection{Vector Addition}
\begin{itemize}
  \item Basic vector addition kernel
\end{itemize}

  \subsection{BigDFT Filter}
\begin{itemize}
  \item Mont-Blanc 2 Application BigDFT kernel
\end{itemize}

\section{Conclusion and Future Work}
\todo{conclusion}

\bibliographystyle{plain}
\bibliography{d5.11}

\end{document}


% vim: set spell ft=tex fo=aw2t expandtab sw=2 tw=100:
