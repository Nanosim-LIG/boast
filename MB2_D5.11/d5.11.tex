\documentclass[11pt, a4paper, twoside]{montblanc2}

\begin{document}
\devnum{[x.x]}
\title{[Deliverable Title]}
\version{[x.x]}
\deadline{[x]}
\level{[See DoW]}
\nature{[See DoW]}
\authors{Name (PARTNER)}
\contributors{Name (PARTNER), Name (PARTNER), Name (PARTNER)}
\reviewers{Name (PARTNER), Name (PARTNER), }
\keywords{[Please list at least three words]}

\maketitle

\begin{changelog}
\change{0.1}{Initial version of the Mont-blanc \LaTeX template}
\change{0.2}{Fixed font colors and headers}
\change{0.3}{Added new \LaTeX commands for commonly used keywords}
\end{changelog}

\frontmatter

\begin{executive}
This document describes how to use the \LaTeX template for Mont-Blanc deliverables.
\end{executive}

\section{Introduction}
Writing deliverables is a tedious task that can easily become a nightmare if you word processor 
starts messing stuff around. Fortunately, almighty Donald Knuth invented \LaTeX to simplify our life 
when reporting to the European Commission. What Dr. Knuth could not figure out was the format our 
belove commissioners like deliverables to be submitted, so he provided the means for us to define 
\LaTeX templates to specify the format of the documents we need to produce. This document presents 
the main features of the Mont-Blanc \LaTeX template for deliverables.

The template here presented is far from being complete, so do not hesitate to contact the author to 
request whatever you might need.

\section{Using the Mont-Blanc \LaTeX Template}
To use the Mont-Blanc \LaTeX template you need to use the \texttt{montblanc2} document class. This 
is done by setting the document header as follows:
\begin{verbatim}
\documentclass[11pt, a4paper, twoside]{montblanc2}
\end{verbatim}

\subsection{Front Page Commands}
The Mont-Blanc \LaTeX template produces a fancy front-page with the project logo, deliverable 
number, version, and name, as well as a document information table. Although many breakthroughs have 
been done in the field of Artificial Intelligence, we have incorporated none of theses in this 
\LaTeX template, so you will need to specify all this information in the top of the document. This 
is done as follows:
\begin{itemize}
\item \verb|\devnum{}|: deliverable number.
\item \verb|\title{}|: deliverable title.
\item \verb|\version{}|: deliverable version number.
\item \verb|\deadline{}|: month number (\eg M6) when the deliverable is due.
\item \verb|\level{}|: dissemination level for the deliverable.
\item \verb|\nature{}|: type of deliverable (\eg Report).
\item \verb|\authors{}|: name of the author. 
\item \verb|\contributors{}|: name of the folks contributing to the deliverable.
\item \verb|\reviewers{}|: names of our beloved reviewers.
\item \verb|\keywords{}|: keywords for the deliverable.
\end{itemize}

After setting all these parameters, the front page is produced with the command
\begin{verbatim}
\maketitle
\end{verbatim}

\subsection{Front Matter Commands}
Deliverables typically have some stuff before the actual meat, namely: change log, table of 
contents, and executive summary. Each of this section can be appended to the document using the 
corresponding environments.

The Change Log environment also defines a specific command (\verb|\change{}{}|) to add changes to 
the document. The Change Log is added to the deliverable as follows:
\begin{verbatim}
\begin{chagelog}
\change{version number}{description of change}
\change{version number}{description of change}
. . .
\end{changelog}
\end{verbatim}


The table of contents is created by including the \verb|\frontmatter| command. Finally, the 
executive summary is included using the \texttt{executive} environment as follows:
\begin{verbatim}
\begin{executive}
Text for the executive summary
\end{executive}
\end{verbatim}

\subsection{Miscellaneous Commands for the Body}
Besides the standard \LaTeX commands, some commodity commands are provided by this template:

\subsubsection{Safe Abbreviations}
Some commonly used abbreviations can easily cause trouble on \LaTeX, so we have included safe 
versions of this abbreviations:
\begin{itemize}
\item \verb|\eg|: \eg
\item \verb|\ie|: \ie
\item \verb|\vs|: \vs
\item \verb|\etc|: \etc
\item \verb|\etal|: \etal
\item \verb|\MB|: \MB
\item \verb|\IO|: \IO
\item \verb|GFW|: \GFW
\end{itemize}

\subsubsection{Annotations}
Quite often we need to include some annotations for others to read during the preparation of a 
deliverable. The following commands have been added to simplify adding such annotations:
\begin{itemize}
\item \verb|\todo{}| is used to indicate that some text is missing and somebody has to add it. This 
command produces text as follows: \todo{This is a TODO example}.
\item \verb|\fixme{}| is used to indicate that some part is wrong and needs to be corrected. This 
command produces text as follows: \fixme{This is a FIXME example}.
\end{itemize}

\section{Reporting Bugs and Wishes}
If you find any bug and\slash or feature that is missing, please report it to 
\texttt{guadalupe.moreno@bsc.es} or \texttt{petar.radojkovic@bsc.es}. We cannot guarantee that the bug\slash feature will get fixed 
immediately, but at least we will know about it.

\end{document}


% vim: set spell ft=tex fo=aw2t expandtab sw=2 tw=100:
