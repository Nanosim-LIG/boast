\subsubsection{Presentation}

BOAST is an automatic performance tuning framework aiming at meta-programming and optimizing computing kernels.
BOAST has been extensively described in the Mont-Blanc 2 deliverable 5.5~\cite{tichadou15} and thus will only be briefly described here.
For a more in-depth presentation reader should refer to the afore mentioned document.

The primary goal of BOAST is to bring performance portability to critical part of high performance computing applications.
Those \emph{computing kernels} are well-defined part of an application that are compute or memory intensive and represent a consequent part of the computing time.
Their inputs are usually limited in complexity and thus those kernels are a prime target for optimization.

Unfortunately, the variety of platforms an HPC code can encounter keeps growing and a lot of time is spent optimizing computing kernels for new architectures.
BOAST attempts to solve this problem by offering the kernel developer several tools:
\begin{itemize}
  \item an Embedded Domain Specific language to describe computing kernels and their possible optimizations,
  \item a code generation engine that can output the kernel in several languages used in the HPC community: C, FORTRAN, CUDA and OpenCL,
  \item a runtime to select the versions to test, build them, execute them and test their performance and accuracy.
\end{itemize}


\subsubsection{Usage}

The usage of BOAST will be illustrated using the following listing.
This example shows how to load BOAST, change it's environment, define a simple vector addition kernel, execute it using different languages and check the accuracy of the obtained results.

\lstset{style=BOAST}
\begin{lstlisting}
require 'BOAST'
include BOAST

set_array_start(0)
set_default_real_size(4)

def vector_add
  n = Int( "n", :dir => :in)
  a = Real("a", :dir => :in,  :dim => [ Dim(n) ] )
  b = Real("b", :dir => :in,  :dim => [ Dim(n) ] )
  c = Real("c", :dir => :out, :dim => [ Dim(n) ] )
  p = Procedure("vector_add", [n,a,b,c]) {
    decl i = Int("i")
    expr = c[i] === a[i] + b[i]
    if (get_lang == CL or get_lang == CUDA) then
      pr i === get_global_id(0)
      pr expr
    else
      pr For(i,0,n-1) {
        pr expr
      }
    end
  }
  return p.ckernel
end

n = 1024*1024
a = NArray.sfloat(n).random
b = NArray.sfloat(n).random
c = NArray.sfloat(n)

epsilon = 10e-15

c_ref = a + b

[FORTRAN, C, CL, CUDA].each { |l|
  set_lang( l )
  puts "#{get_lang_name}:"
  k = vector_add
  puts k.print
  c.random!
  k.run(n, a, b, c, :global_work_size => [n,1,1], :local_work_size => [32,1,1])
  diff = (c_ref - c).abs
  diff.each { |elem|
    raise "Warning: residue too big: #{elem}" if elem > epsilon
  }
}
puts "Success!"
\end{lstlisting}

\paragraph{Loading BOAST} is done on lines 1 and 2.
It loads the BOAST framework in its environmental configuration.
This configuration can be changed through configuration files or environment variables.
Here, this configuration is changed on lines 4 and 5 to set the default real variable size to 4 bytes and the first index of arrays to 0 (C like arrays).

\paragraph{Kernel definition} is done on lines 7 to 25.
The definition is enclosed in a procedure that will be called in different context yielding different kernel versions.
First the parameters of the kernel are defined on lines 8 to 11, and stored in ruby variables.
Those parameters have a direction (similar to FORTRAN intent) and the \emph{Real} parameters have a dimension meaning they are arrays, in this case of length \emph{n}.
The \emph{Procedure} is declared on line 12 by giving its name and parameter list.
It is associated a block of code that spans from line 12 to 23 and constitute its body.
A local integer variable \emph{i} is defined on line 12 and declared using the keyword \emph{decl}.
On line 14, an affectation expression is saved for later use in the ruby variable \emph{expr}.
This expression computes one element of the result array.
The difference between a ruby affectaion (using the \emph{=} operator) and a BOAST affectation (using the \emph{===} operator) has to be noted.
On lines 15 to 22 we can see some BOAST meta-programming.
Here, based on the BOAST \emph{lang} state, the code will select either a GPU or CPU implementation.
The GPU implementation obtains an index of element to process (line 16) and stores it into the \emph{i} variable.
This is printed using the \emph{pr} keyword, meaning that this line is to appear in the generated source.
The expression we saved earlier is then printed (line 17).
The CPU implementation prints a for loop that will process all the elements of the arrays.
We can see here an example, though far-fetched, of code factoring.
